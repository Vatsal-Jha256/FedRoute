% TikZ Flowcharts for FedRoute Paper
% Include these in your LaTeX paper

% Required packages:
% \usepackage{tikz}
% \usetikzlibrary{shapes.geometric, arrows, positioning, fit, calc}

% ==================================================================
% FLOWCHART 1: FedRoute System Architecture
% ==================================================================

\begin{figure}[ht]
\centering
\begin{tikzpicture}[
    node distance=1.2cm and 2cm,
    every node/.style={font=\small},
    component/.style={rectangle, draw, fill=blue!10, text width=3.5cm, text centered, rounded corners, minimum height=1cm},
    data/.style={cylinder, shape border rotate=90, draw, fill=green!10, text width=2cm, text centered, minimum height=0.8cm},
    process/.style={rectangle, draw, fill=orange!10, text width=3cm, text centered, rounded corners, minimum height=0.9cm},
    decision/.style={diamond, draw, fill=yellow!10, text width=2.2cm, text centered, aspect=2},
    arrow/.style={thick,->,>=stealth}
]

% Top layer - Data Sources
\node[data] (mobility) {Vehicle Trajectories};
\node[data, right=of mobility] (music) {Music Listening Data};
\node[data, right=of music] (pois) {POI Database};

% Middle layer - Client Processing
\node[component, below=1.5cm of music] (client) {Client Vehicles\\(Edge Devices)};
\node[process, below=of client] (local) {Local FMTL\\Training};

% Feature extraction
\node[process, left=2.5cm of local] (features) {Context\\Extraction};

% Server layer
\node[component, below=1.8cm of local] (server) {Federated Server};
\node[process, below=of server] (agg) {Secure\\Aggregation};

% Selection
\node[decision, right=2cm of server] (selection) {Client\\Selection};

% Model components
\node[component, below=1.5cm of agg] (encoder) {Shared Context Encoder};
\node[process, below left=0.8cm and 0.5cm of encoder] (path) {Path Head};
\node[process, below right=0.8cm and 0.5cm of encoder] (musichead) {Music Head};

% Privacy
\node[process, left=2.5cm of agg, fill=red!10] (privacy) {Differential\\Privacy};

% Arrows
\draw[arrow] (mobility) -- (client);
\draw[arrow] (music) -- (client);
\draw[arrow] (pois) -- (client);
\draw[arrow] (client) -- (local);
\draw[arrow] (local) -- (server);
\draw[arrow] (server) -- (agg);
\draw[arrow] (agg) -- (encoder);
\draw[arrow] (encoder) -- (path);
\draw[arrow] (encoder) -- (musichead);
\draw[arrow] (selection) -- (server);
\draw[arrow] (privacy) -- (agg);
\draw[arrow] (features) -- (local);
\draw[arrow] (client) -- (features);

% Global model update
\draw[arrow, dashed, bend left=45] (encoder) to node[right, font=\footnotesize] {Global\\Model} (server);

\end{tikzpicture}
\caption{FedRoute System Architecture: The framework integrates vehicle trajectories, music listening patterns, and POI data through a federated multi-task learning approach with privacy-preserving mechanisms.}
\label{fig:system_architecture}
\end{figure}

% ==================================================================
% FLOWCHART 2: Federated Learning Training Process
% ==================================================================

\begin{figure}[ht]
\centering
\begin{tikzpicture}[
    node distance=1cm and 1.5cm,
    every node/.style={font=\small},
    start/.style={ellipse, draw, fill=green!20, text width=2cm, text centered, minimum height=0.8cm},
    process/.style={rectangle, draw, fill=blue!10, text width=3.2cm, text centered, rounded corners, minimum height=0.9cm},
    decision/.style={diamond, draw, fill=yellow!10, text width=2.5cm, text centered, aspect=2.5},
    end/.style={ellipse, draw, fill=red!20, text width=2cm, text centered, minimum height=0.8cm},
    arrow/.style={thick,->,>=stealth}
]

% Start
\node[start] (start) {Initialize Global Model};

% Main process flow
\node[process, below=of start] (round) {Round $t$};
\node[process, below=of round] (select) {Multi-Objective Client Selection};
\node[process, below=of select] (broadcast) {Broadcast Global Model $w^t$};
\node[process, below=of broadcast] (local1) {Local Training:\\$K$ epochs on each client};
\node[process, below=of local1] (clip) {Gradient Clipping\\(DP mechanism)};
\node[process, below=of clip] (upload) {Upload Local Updates};
\node[process, below=of upload] (aggregate) {Secure Aggregation\\+ DP Noise};
\node[process, below=of aggregate] (update) {Update Global Model:\\$w^{t+1}$};

% Decision
\node[decision, below=1.2cm of update] (converged) {Converged or\\Max Rounds?};

% End
\node[end, below=1cm of converged] (end) {Final Model};

% Loop back
\coordinate[left=3cm of round] (loop);

% Arrows
\draw[arrow] (start) -- (round);
\draw[arrow] (round) -- (select);
\draw[arrow] (select) -- (broadcast);
\draw[arrow] (broadcast) -- (local1);
\draw[arrow] (local1) -- (clip);
\draw[arrow] (clip) -- (upload);
\draw[arrow] (upload) -- (aggregate);
\draw[arrow] (aggregate) -- (update);
\draw[arrow] (update) -- (converged);
\draw[arrow] (converged) -- node[right] {Yes} (end);
\draw[arrow] (converged) -- node[above] {No} ++(-2.5,0) |- (loop) -- (round);

% Annotations
\node[right=1.5cm of select, text width=3cm, font=\footnotesize, align=left] {
    \textbf{Objectives:}\\
    • Accuracy\\
    • Efficiency\\
    • Fairness\\
    • Privacy
};

\node[right=1.5cm of aggregate, text width=3cm, font=\footnotesize, align=left] {
    \textbf{Privacy:}\\
    • $(\epsilon, \delta)$-DP\\
    • $\epsilon = 1.0$\\
    • $\delta = 10^{-5}$
};

\end{tikzpicture}
\caption{Federated Learning Training Process: FedRoute employs multi-objective client selection, local FMTL training with differential privacy, and secure aggregation to produce a globally optimized model while preserving user privacy.}
\label{fig:fl_process}
\end{figure}

% ==================================================================
% Usage instructions:
% 1. Copy the desired flowchart into your LaTeX document
% 2. Make sure you have the required packages in your preamble
% 3. Adjust node positioning as needed for your paper layout
% 4. Customize colors and styles to match your paper's theme
% ==================================================================


